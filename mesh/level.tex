\magnification=\magstep1
\def\IMSL{{\tt IMSL}}
\def\today{\ifcase\month\or
  January\or February\or March\or April\or May\or June\or
  July\or August\or September\or October\or November\or December\fi
  \space\number\day, \number\year}

\centerline{\bf Code notes for {\tt level.f}}
\medskip
\centerline{S. Scott Collis}
\smallskip
\centerline{\today}
\bigskip

{\tt level.f} does a least-squares B-spline fit using fifth order B-splines
(fourth order polynomials) and builds a body-fitted mesh with algebraic
mapping functions in $s$ and $n$.  This code was developed to generate meshes
for Howard Kanner's airfoil, but other geometries can also be used.

First, the arc-length, $s$, along the surface is estimated using piecewise
linear interpolation and this estimate is used as the absissca for the
B-spline interpolation.  I use the \IMSL\ routine {\tt BSVLS} to compute the
optimal knot sequence individually for both $x$ and $y$ on the boundary.
Then, the \IMSL\ routine {\tt CONFT}, which performs a least-squares B-spline
with linear constraints, is called to determine a smooth B-spline interpolant
which enforces symmetry at the leading edge.  I have found from
experimentation that $k=5$, $n=13$ yields a reasonable approximation to
Horward's rough points.  A slightly worse fit is given by $k=5$, $n=10$,
however the normal vectors (i.e. the curvature) are a little better behaved.
If need be, I think that one could use these values for a better conditioned
curve, but I'll start with the $n=13$ value first.

The smoothed B-spline and knot sequence are saved in the file {\tt bs.dat}.
The arclength along the B-spline is computed using RK4 integration and by
using the routine {\tt BSDER} to compute the derivative of the B-spline
representation.  Various tangent and normal mapping functions are available by
setting command line arguments.  To see a complete list of available options,
type {\tt level -h}.  Given the B-spline representation, level sets are
computed using the derivative of the B-spline to determine the surface normal
vector.  Finally, there is the option of performing an elliptic grid smoothing
and the final grid is saved in {\tt Plot3d} format.

\bye




